% !TEX root = ../my-thesis.tex
%
\pdfbookmark[0]{Acknowledgement}{Acknowledgement}
\addchap*{Acknowledgement}
\label{sec:acknowledgement}

A PhD, in which you have to do a research project, is a daunting task. How could I possibly frame the questions
that would lead to significant discoveries; design and interpret an experiment so that the conclusions were
absolutely convincing; foresee difficulties and see ways around them, or, failing that, solve them when they
occurred?

These where the reasons I didn't even apply for PhD projects after my masters in Mathematics and rather decided
to wander around the Himalayas. The seeds of this PhD were sown then, during my 3-year long immersion with the
mountain communities of Ladakh. Living among mountains of sand, I recognised the futility of human life but was
also alarmed by how small human-scale actions could transform these mountains beyond recognition. How is this
even possible and what can be done about it? Looking back, those were the questions that pushed me to pursue
this journey. 

But why did I end up here in Switzerland, my first foreign country and how did I get an opportunity to pursue a
PhD in the domain of glaciology, a subject which I had no knowledge about? Short answer: The trust and support
of three people: Sonam Wangchuk, Felix Keller, and Martin Hoelzle. With Sonam, there were always a plethora of
great ideas around. All I had to do was pick my favourite and pursue it. With Felix, Switzerland became a
country where my second home is. In Martin, I found a colleague twice my age whose engagement around scientific
questions was unmatched. I owe a lot to each of them who mentored me around in Phyang, Samedan and Fribourg for
the past six years.

A PhD thesis does not write itself, nor does its author operate from a deserted island. As such, I would like to
acknowledge the many individuals that contributed to the realization of this work in, what will most likely be,
one of the most read sections of this thesis. This work would not have been possible without the support,
collaboration and friendship of my family, friends and colleagues. To that end, I attempted to thank all of them
in the following paragraphs.

First and foremost, words cannot express my gratitude and appreciation for professor Martin Hoelzle, my mentor.
You initiated and guided me through the wondrous world of science, at every step giving me that extra push to
believe in myself. I am grateful for the entire journey; from writing proposals and manuscripts to suffering
through the destructive feedbacks together, from discussing research problems to helping others write their
bachelors and masters thesis around them, from designing experiments on the blackboard to establishing Swiss and
Indian icestupa laboratories, from mere equations to real insights, to today, almost holding my PhD.

My Ph.D. project was somewhat interdisciplinary and, for a while, whenever I ran into a problem, I pestered a
lot of people for help: from the univerity concierge Tony for his experience with pipelines to the faculty who
were experts in the various disciplines that I needed. I remember the day when Johannes Oerlemans (who won the
International Glaciological Society's Richardson medal two years later) told me he didn't know how to solve the
problem I was having in his area. I was a first-year graduate student and I figured that Oerlemans knew much
more than I did. If he didn't have the answer, nobody did.

That's when it hit me: nobody did. That's why it was a research problem. And being my research problem, it was
up to me to solve. Once I faced that fact, the going got easy. The crucial lesson was that the scope of things I
didn't know wasn't merely vast; it was, for all practical purposes, infinite. That realization, instead of being
discouraging, was liberating. If our ignorance is infinite, the only possible course of action is to muddle
through as best we can.

Last but not the least, I would like to express the most sincere gratitude to my parents. Some of my hard life
choices like volunteering in the mountains or shifting abroad would have been impossible if it weren't for the
financial and emotional support assured from my parents if things fell apart.


% That is how I got the confidence, perseverance and
% commitment needed to see through this journey in a foreign country with almost no subject knowledge to lean back
% on.  

% The Indian part of this journey India 

% guiding bachelors and masters students to 

% found a renewed appreciation for the societal
% transformation possible with science and technology. 

% to reviewing manuscripts 

% Well, to say this was my thesis would be totally untrue. At best, this was my dream. THere are people in this
% world, some of them so wonderful, that made this dream become a product that you are looking at. I would like to
% thank all of them, and in particular:

% Martin Hoelzle found means to employ me at UniFR 10 months before the actual takeoff of the PhD project. 

% Dozens of people have contributed with days and days of work to compiling this data basis.


% Eluckkiya dragged me out of the office.

% My parents --

% Anupama Pain --

% Sonam Wangchuk --

% Felix Keller --

% Ladakh colleagues --


% Staying in Ladakh was also a spiritual experience. The immensity of the mountains belittles your existence.
% Living among mountains of sand, it was easy to question the impact one can have.
% I often wonder wherefrom the confidence, perseverance and commitment needed to see through this PhD came from. 

