% !TEX root = ../my-thesis.tex
%
\pdfbookmark[0]{Acknowledgement}{Acknowledgement}
\addchap*{Acknowledgement}
\label{sec:acknowledgement}

The seeds of this PhD were sown during my 3-year long immersion with the mountain communities of Ladakh. It was
there that I found a renewed appreciation for the societal transformation possible with science and technology
(Fig. ). Living among mountains of sand, it was easy to question the impact one can have. 

I often wonder wherefrom the confidence, perseverance and commitment needed to see through this PhD came from. 

% If it wasn't for my 3-year long immersion with the mountain communities of Ladakh, I would never have had the
% perseverance and commitment necessary to see through this journey. If it wasn't for th

% A PhD thesis does not write itself, nor does its author operate from a deserted island. As such, I would like to
% acknowledge the many individuals that contributed to the realization of this work in, what will most likely be,
% one of the most read sections of this thesis. This work would not have been possible without the support,
% collaboration and friendship of my family, friends and colleagues during the past 4 years. To that end, I
% attempted to thank all of them in the following paragraphs.

% First and foremost, words cannot express my gratitude and appreciation for professor Martin Hoelzle, my mentor.
% You initiated and guided me through the wondrous world of science, at every step giving me that exta push to
% believe in myself. I am grateful for the entire journey; from writing PhD proposals to suffering through the
% destructive feedbacks together, from designing experiments on the blackboard to establishing Swiss and Indian
% icestupa laboratories, from mere equations to real insights, to today, almost holding my PhD. 

% guiding bachelors and masters students to 

% to reviewing manuscripts 

% My Ph.D. project was somewhat interdisciplinary and, for a while, whenever I ran into a problem,
% I pestered the faculty in my department who were experts in the various disciplines that I needed. I remember
% the day when Henry Taube (who won the Nobel Prize two years later) told me he didn't know how to solve the
% problem I was having in his area. I was a third-year graduate student and I figured that Taube knew about 1000
% times more than I did (conservative estimate). If he didn't have the answer, nobody did.

% That's when it hit me: nobody did. That's why it was a research problem. And being my research problem, it was
% up to me to solve. Once I faced that fact, I solved the problem in a couple of days. (It wasn't really very
% hard; I just had to try a few things.) The crucial lesson was that the scope of things I didn't know wasn't
% merely vast; it was, for all practical purposes, infinite. That realization, instead of being discouraging, was
% liberating. If our ignorance is infinite, the only possible course of action is to muddle through as best we
% can.

% Well, to say this was my thesis would be totally untrue. At best, this was my dream. THere are people in this
% world, some of them so wonderful, that made this dream become a product that you are looking at. I would like to
% thank all of them, and in particular:

% Martin Hoelzle found means to employ me at UniFR 10 months before the actual takeoff of the PhD project. 

% Dozens of people have contributed with days and days of work to compiling this data basis.

% This is also the place to express the most sincere gratitude to my parents. They have supported me throughout
% all the years of my educattion in many different ways and have made it possible to do what I really wanted to.  

% Eluckkiya dragged me out of the office.


% Staying in Ladakh was a spiritual experience. The immensity of the mountains belittles your existence.

% My parents --

% Anupama Pain --

% Sonam Wangchuk --

% Felix Keller --

% Ladakh colleagues --

% The number of things that had to happen for me to take such a leap of faith still overwhelm me.

% A PhD, in which you have to do a research project, is a daunting task. How could I possibly frame the
% questions that would lead to significant discoveries; design and interpret an experiment so that the conclusions
% were absolutely convincing; foresee difficulties and see ways around them, or, failing that, solve them when
% they occurred? 
