% !TEX root = ../my-thesis.tex
%
\pdfbookmark[0]{Acknowledgement}{Acknowledgement}
\addchap*{Acknowledgement}
\label{sec:acknowledgement}

A PhD, in which you have to do a research project, is a daunting task. How could I possibly frame the questions
that would lead to significant discoveries; design and interpret an experiment so that the conclusions were
absolutely convincing; foresee difficulties and see ways around them, or, failing that, solve them when they
occurred?

These where the reasons I didn't even apply for PhD projects after my masters in Mathematics and rather decided
to wander around the Himalayas. The seeds of this PhD were sown then, during my 3-year long immersion with the
mountain communities of Ladakh. Living among mountains of sand, I recognised the futility of human life but was
also alarmed by how small human-scale actions could transform these mountains beyond recognition. How is this
even possible and what can be done about it? Looking back, those were the questions that pushed me to pursue
this journey. 

But why did I end up here in Switzerland, my first foreign country and how did I get an opportunity to pursue a
PhD in the domain of glaciology, a subject which I had no knowledge about? Short answer: The trust and support
of three people: Sonam Wangchuk, Felix Keller, and Martin Hoelzle. With Sonam, there were always great ideas
around. All I had to do was pick my favourite and pursue it. With Felix, Switzerland became a country where my
second home is. In Martin, I found a colleague twice my age whose engagement around scientific questions was
unmatched. I owe a lot to each of them who mentored me around in Phyang, Samedan and Fribourg for the past six
years.

A PhD thesis does not write itself, nor does its author operate from a deserted island. As such, I would like to
acknowledge the many individuals that contributed to the realization of this work in, what will most likely be,
one of the most read sections of this thesis. This work would not have been possible without the support,
collaboration and friendship of my family, friends and colleagues. To that end, I attempted to thank all of them
in the following paragraphs.

First and foremost, words cannot express my gratitude and appreciation for professor Martin Hoelzle, my mentor.
You initiated and guided me through the wondrous world of science, at every step giving me that extra push to
believe in myself. I am grateful for the entire journey; from writing proposals and manuscripts to suffering
through the destructive feedbacks together, from discussing research problems to helping others write their
bachelors and masters thesis around them, from designing experiments on the blackboard to establishing Swiss and
Indian icestupa laboratories, from mere equations to real insights, to today, almost holding my PhD.

My Ph.D. project was somewhat interdisciplinary and, for a while, whenever I ran into a problem, I pestered a
lot of people for help: from the univerity concierge Tony for his experience with pipelines to the faculty who
were experts in the various disciplines that I needed. I remember the day when Johannes Oerlemans (who won the
International Glaciological Society's Richardson medal two years later) told me he didn't know how to solve the
problem I was having in his area. I was a first-year graduate student and I figured that Oerlemans knew much
more than I did. If he didn't have the answer, nobody did.

That's when it hit me: nobody did. That's why it was a research problem. And being my research problem, it was
up to me to solve. Once I faced that fact, the going got easy. The crucial lesson was that the scope of things I
didn't know wasn't merely vast; it was, for all practical purposes, infinite. That realization, instead of being
discouraging, was liberating. If our ignorance is infinite, the only possible course of action is to muddle
through as best we can.

Without the generous support from the Swiss Government Excellence Scholarship (SGES) and the University of
Fribourg (UniFR), this research would not have been possible. Beyond financial support, the SGES were also very
accomodating for my field work requirements during a global pandemic and made exceptions for me that have never
been made before. I am also grateful to have received grants from the Swiss Polar Institute and GlaciersAlive
Association that enabled the extensive fieldwork over the past few winters. 

Dozens of people have contributed with days and days of work for collecting the associated datasets. My special
thanks goes to the Himalayan Institute of Alternatives, Ladakh (HIAL) team who provided administrative and
material support to conduct field campaigns in India. Norboo Thinles, Nishant Tiku, Sourabh Maheshwari and Dr.
Tom Matthews were instrumental for making the Ladakh measurement campaign a success. Without the unconditional
support of Daniel Bürki, the frequent flights and expensive installations in Guttannen would not have been
possible. I would also like to thank Adolf Kaeser and Mr. Flavio Catillaz from Eispalast Schwarzsee (CH19). Even
though no datasets from CH19 AIR appear in the thesis or our publications, the field experience we gained their
was a necessary precondition for success in our future sites. 

One of the main reasons I arrived (almost) everyday with a big smile to work are the fabulous colleagues of the
Cryosphere department, many of who became good friends. My office mates over the years, Shafaq, Mario, Ottavia,
Romain, Rebecca, and Esther made our room quite lively, which I missed a lot during periods of mandatory remote
work. These colleagues coming from literally all parts of the world, gave me an incredible international
experience. It was also a great experience to teach Bachelors courses with Horst and Eric. Luke deserves a
special mention for the one with the most wisdom in the department. His timely recommendation to use the
Schwarzsee Eispalast as my first experiment site kickstarted my PhD in earnest. I also like to thank the
administrative staff for welcoming me into the department. David and Sylvie, apart from taking care of my
logistic and financial hurdles, also nudged me in every encounter to become a better french speaker. Nicole, for
taking that extra care towards sorting all my visa anxieties. Alex showed me how cool technicians can be with
his stories and youthfulness. I feel very blessed to have spent time with all of you, ranging from many
afterwork drinks, numerous coffee breaks, lunches, dinners and BBQs, conferences, train rides, and treks. All of
you contributed to the exceptional warm and cosy atmosphere at our department and I cherished every minute. 

Not everything is science! Eluckkiya made sure it was that way. Without her, I feel I would have languished much
longer within these four walls.

Last but not the least, I would like to express the most sincere gratitude to my parents. Some of my hard life
choices like volunteering in the mountains or shifting abroad were worth considering only because I always had
their emotional and financial support to lean back on.

Michelle Stirnimann and Maria Gruber

