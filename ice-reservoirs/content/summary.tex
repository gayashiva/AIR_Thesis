% \pdfbookmark[0]{Summary}{Summary}
\addchap{Summary}
\label{sec:summary}

Irrigated agriculture is crucial for the livelihood security of mountain communities. Using meltwater from
glaciers, snow and permafrost, mountain dwellers have developed sophisticated techniques to cope with recurrent
water scarcity caused by glacial retreat and seasonal snow-cover dynamics. Artificial ice
reservoirs (AIRs) are a key example of community-based water management. Worldwide, farmers in around 30 mountain
villages build these ice structures. These seasonal ice reservoirs increase meltwater availability during the
critical period of water scarcity. To assess the role of AIRs within the water resource management of
mountain villages under a changing climate, they need to be represented in integrated modelling frameworks.
Efficient water storage can be achieved by taking into account their local meteorological conditions and
water availability throughout the year. This thesis aims to increase the understanding of volume dynamics of
 \ac{AIRs} to provide tools to reduce their water losses and maintenance requirements.

The different contributions of surface processes of the \ac{AIRs} built in Guttannen, Switzerland and  Gangles, India were estimated. These two
locations present different meteorological patterns due to their significant latitudinal and altitudinal
differences. Using \ac{AIR}-specific mass and energy balance models that consider meteorological factors,
fountain discharge, and ice volume changes, surface processes were quantified and compared across the two
locations. The models successfully estimated the observed ice volume evolution with a root mean square
error within 20\% of the maximum ice volume for five \ac{AIRs}. The location in Ladakh presented a maximum ice volume four
times larger than that of the Guttannen site. However, poor fountain operation resulted in wastage of more than
four fifths of the provided water supply. These results highlight the relevance of colder drier climates and
fountain water supply management in the optimization of \ac{AIR} construction.

In addition, \ac{AIR} water loss reduction was attempted on the Swiss Alps by implementing fountain scheduling
strategies. Fountain scheduling was performed using a ball valve automated with optimal discharge
rates computed using customized glacial models. Simulations converting nonscheduled fountains into scheduled
fountains showed a more than threefold improvement in the water use efficiency of several \ac{AIRs}. Fountain
operation using scheduling strategies produced similar ice volumes while consuming one tenth of the water
compared with their nonscheduled counterparts. Overall, these results show that automated fountain water supply
management can both increase the water-use efficiency of \ac{AIRs} and reduce their maintenance needs without
compromising their meltwater production.

This thesis introduces, for the first time, a model- and measurement-based understanding on the volume evolution of
\ac{AIRs} under different climates; it also presents methods to quantify the storage potential of these ice structures
worldwide and practical tools to improve their efficacy. This study improves the scientific evidence needed to
upscale this indigenous water storage technology. These findings are essential to design these nature-based
solutions that increase the reliability of water supply in highly seasonal and arid environments and improve
water security and climate change adaptation in mountain regions. Future work may build on this research by
fully integrating climate change scenarios to investigate the potential hydrological contributions of ice
harvesting technologies for water-stressed mountain catchments.
