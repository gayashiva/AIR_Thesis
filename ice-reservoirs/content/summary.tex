% \pdfbookmark[0]{Summary}{Summary}
\addchap{Summary}
\label{sec:summary}

Irrigated agriculture is crucial for the livelihood security of mountain communities. Using meltwater from
glaciers, snow and permafrost, mountain dwellers have developed sophisticated techniques to cope with recurrent
water scarcity caused by glacial retreat and seasonal snow-cover dynamics. Artificial ice reservoirs (AIRs) are
a key example of community-based water management. Worldwide, farmers in around 30 mountain villages build these
ice structures. These seasonal ice reservoirs increase meltwater availability during the critical period of
water scarcity. To assess the role of AIRs within the water resource management of mountain villages under a
changing climate, they need to be represented in integrated modelling frameworks. Efficient water storage can be
achieved by taking into account their local meteorological conditions and water availability throughout the
year. This thesis aims to increase the understanding of volume dynamics of AIRs to provide tools to reduce their
water losses and maintenance requirements.

The different contributions of surface processes of the AIRs built in Guttannen, Switzerland and  Gangles, India
were estimated. These two locations present different meteorological patterns due to their significant
latitudinal and altitudinal differences. Using AIR-specific mass and energy balance models that consider
meteorological factors, fountain discharge, and ice volume changes, surface processes were quantified and
compared across the two locations. The models successfully estimated the observed ice volume evolution with a
root mean square error within 20\% of the maximum ice volume for five AIRs. The location in Ladakh presented a
maximum ice volume four times larger than that of the Guttannen site. However, poor fountain operation resulted
in wastage of more than four fifths of the provided water supply. These results highlight the relevance of
colder drier climates and fountain water supply management in the optimization of AIR construction.

In addition, AIR water loss reduction was attempted on the Swiss Alps by implementing fountain scheduling
strategies. Fountain scheduling was performed using a ball valve automated with optimal discharge rates computed
using AIR models. Simulations converting nonscheduled fountains into scheduled fountains showed a more than
threefold improvement in the water use efficiency of several AIRs. Fountain operation using scheduling
strategies produced similar ice volumes while consuming one tenth of the water compared with their nonscheduled
counterparts. Overall, these results show that automated fountain water supply management can both increase the
water-use efficiency of AIRs and reduce their maintenance needs without compromising their meltwater production.

This thesis introduces, for the first time, a model- and measurement-based understanding on the volume evolution
of AIRs under different climates; it also presents methods to quantify the storage potential of these ice
structures worldwide and practical tools to improve their efficacy. This study improves the scientific evidence
needed to upscale this indigenous water storage technology. These findings are essential to design these
nature-based solutions that increase the reliability of water supply in highly seasonal and arid environments
and improve water security and climate change adaptation in mountain regions. Future work may build on this
research by fully integrating climate change scenarios to investigate the potential hydrological contributions
of ice harvesting technologies for water-stressed mountain catchments.

\addchap{Résumé}

L'agriculture irriguée est cruciale pour sécuriser les moyens de subsistance des communautés de montagne. En utilisant les eaux issues de la fonte des glaciers, de la neige et du pergélisol, les habitants des montagnes ont développé des techniques sophistiquées pour faire face à la pénurie d'eau récurrente causée par le retrait des glaciers et la variabilité saisonnière de la couverture neigeuse. Les réservoirs de glace artificiels (RIA) sont un exemple clé de gestion communautaire de l'eau. Dans le monde entier, les agriculteurs d'une trentaine de villages de montagne construisent ces structures de glace. Ces réservoirs de glace saisonniers augmentent la quantité d’eau de fonte disponible pendant la période critique de pénurie d'eau. Pour évaluer le rôle des RIA au sein de la gestion des ressources en eau des villages de montagne dans un contexte de changement climatique, il est nécessaire de les représenter dans des cadres de modélisation intégrés. En tenant compte des conditions météorologiques locales et de la disponibilité de l'eau tout au long de l'année, l'eau peut être efficacement stockée. Cette thèse vise à améliorer la compréhension de la dynamique des volumes des RIA afin de fournir des outils pour réduire leurs pertes d'eau et besoins d'entretien. 

Les différentes contributions des processus de surface ont été estimées pour des RIA construits à Guttannen, en Suisse, et à Gangles, au Ladakh en Inde. Ces deux sites sont affectés par des conditions météorologiques distinctes en raison de leurs importantes différences latitudinales et altitudinales. À l'aide de modèles de bilan de masse et d'énergie spécifiques aux RIAs qui prennent en compte les facteurs météorologiques, le débit des fontaines et les changements de volume de glace, les processus de surface ont été quantifiés et comparés entre les deux sites. Les modèles ont reproduit avec succès l'évolution observée du volume de glace de cinq RIA avec une erreur quadratique moyenne de 20 \% du volume maximal de glace. Le site du Ladakh présentait un volume maximal de glace quatre fois supérieur à celui du site de Guttannen. Cependant, le mauvais fonctionnement de la fontaine a entraîné un gaspillage de plus de quatre cinquièmes de la quantité d'eau disponible. Ces résultats soulignent l'importance des climats plus froids et plus secs et de la gestion de l'approvisionnement en eau des fontaines dans l'optimisation de la construction des RIAs. 

En outre, grâce à différentes stratégies de programmation des fontaines, des essais de réduction des pertes d'eau ont été effectué dans les Alpes suisses. La programmation des fontaines a été réalisée à l'aide d'une vanne à bille automatisée permettant l’ajustement du débit aux valeurs optimales calculées à l'aide des modèles de RIA. Des simulations comparant des fontaines non programmées avec des fontaines programmées ont montré une amélioration de plus de trois fois de l'efficacité de l'utilisation de l'eau de plusieurs RIAs. Le fonctionnement des fontaines utilisant des stratégies de programmation a produit des volumes de glace similaires tout en consommant un dixième de la quantité d’eau de leurs homologues non programmées. Dans l'ensemble, ces résultats montrent que la gestion automatisée de l'approvisionnement en eau des fontaines peut à la fois augmenter l'efficacité de l'utilisation de l'eau des RIAs et réduire leurs besoins de maintenance sans compromettre leur production d'eau de fonte.

Cette thèse présente, pour la première fois, une compréhension de l'évolution du volume des RIAs sous différents climats sur la base de modèles et de mesures; elle présente également des méthodes pour quantifier le potentiel de stockage de ces structures de glace dans le monde entier et des outils pratiques pour améliorer leur efficacité. Cette étude améliore les connaissances scientifiques nécessaires à l'expansion de cette technologie indigène de stockage de l'eau. Ces résultats sont essentiels à la conception de solutions naturelles augmentant la fiabilité de l'approvisionnement en eau dans les environnements arides dépendant de source d’eau saisonnières. Ils contribuent à l’amélioration de la sécurité de l’approvisionnement en eau et à l'adaptation au changement climatique des régions de montagnes. De futurs travaux pourraient s'appuyer sur cette recherche en intégrant pleinement les scénarios de changement climatique afin d’étudier les potentielles contributions hydrologiques de ces technologies de récolte de la glace au niveau des bassins versants soumis à un stress hydrique.