\chapter{Religion of ice reservoirs}

\cleanchapterquote{We believe that glaciers are alive. That's why a combination of
female and male ice was necessary.}{Liaquat Ali Baltee}{(Resident of Skardu, Pakistan)}

For centuries, in the Himalayan mountain ranges, local cultures have believed that glaciers are alive and that
certain glaciers can also have different genders. These local communities ‘breed’ new glaciers by grafting
together —or marrying— fragments of ice from male and female glaciers covering them with charcoal, wheat husks,
cloths, or willow branches so they can reproduce in privacy. These glacierets transform into fully active
glaciers by growing year by year with additional snowfall, serving as lasting reservoirs of water that allow
farmers to irrigate their crops. Over time, these practices have inspired other cultures, in which people are
now creating their own \ac{AIRs} and using them to solve urgent challenges around water supplies.

\section{An old story}

According to legend, when the people of Baltistan, in Pakistan, learnt of the Mongol army advancing towards them
from the north in the early $13^{th}$ century, they came up with an ingenious way to stop them. As the inhabited
valleys were only accessible through narrow passes, they decided to block the entry way by building a glacier.
This successfully prevented the Mongol invasion and, crucially, also solved the locals’ other big problem: water
scarcity \citep{khanMarriageGlaciersPrzekroj2020}.

\section{The marriage of glaciers}

The people of Gilgit Baltistan believe that glaciers are living entities
\citep{farazGlacierMarriagesPakistan2020, khanMarriageGlaciersPrzekroj2020}. Therefore, a combination of female
and male ice is absolutely necessary for them to multiply and grow. The male glacier –locally called ‘po gang’–
gives off little water and moves slowly, while a ‘female glacier’ –or ‘mo gang’– is a growing glacier that gives
off a lot of water. 

The glaciers that people help grow are the fruit of the sacred union between a mother glacier and a father
glacier. The ice formations get married and produce offspring. For local communities, the selection of an
appropriate site for this marriage is of utmost importance, and a suitable site must fulfil a list of
conditions. It should be located at an altitude of at least 4000 to 5000 m \ac{a.s.l.} and should be on a gentle
slope with minimal exposure to sunlight, thus a north-facing mountain side is preferable. For most expert
glacier grafters, the presence of permafrost or ice on the site is another key requirement. 

Once a suitable spot is selected, the expedition can be planned. The bride and groom---the female and the male
glacier, preferably from different villages---are chosen, and the marriage can be organized. The glacier
grafting usually takes place in November, when the local temperatures oscillate around zero. The process used to
conduct this glacier marriage ceremony is described in the Appendix \ref{sec:glacier_marriage}.

\section{From folklore to science}

Myths, legends, and superstitions are ways of codifying and disseminating knowledge. However, in the face of a
mounting climate crisis, they now need to be translated into the language of science. 

Classifying glaciers as male and female is, of course, a practice motivated by deep religious beliefs, but the
method used to achieve this classification hints at an even deeper understanding of their temporal discharge
patterns. In scientific terms, male glaciers are those which have achieved their peak water, likely leading to
imminent water scarcity in their catchment. The concept of 'peak water' implies that, first, as glaciers shrink
in response to a warmer climate, more meltwater is released until a turning point (peak water), after which
glaciers melt, and so its contribution to river flow decreases. Based on this perspective, accelerated glacier
shrinkage due to climate change is causing a gender imbalance. Such narratives are not designed to stand up to
scientific scrutiny but rather to illustrate the state of the world in the most simple and effective way that
can inspire societal change. 

However, when it comes to glacier building expeditions, the evidence available is scant and anecdotal. According
to \citet{tveitenGlacierGrowingLocal2007}, the account of the glacier development process presented by a glacier
grafter from the Baltistan region bears a strong resemblance to the definition of the formation of rock
glaciers. \citet{tveitenGlacierGrowingLocal2007} concludes that: 

\begin{thesis_quotation}

“glacier growing is typically performed […] in a terrain that is conducive to the accumulation of snow by
avalanching and snow slips. The presence of permafrost at these locations is likely to contribute to ice
accumulating […]. Thus, glacier growing is conducted at locations which are already very prone to ice
accumulation, and may explain why glacier growing is perceived to work.” 

\end{thesis_quotation}

Therefore, the practices involved in glacier grafting do not have scientific evidence to support their efficacy
in building artificial glaciers. Still, they have been an effective tool to communicate the effects of glacier
shrinkage and instigate action for their preservation across regional and global scales. 

The view of glaciers as animate entities implies that humans can influence their lives, just as glaciers can
influence the lives of people. A fact that much of the world is still catching up to.

These mythologies and practices from Baltistan, Pakistan inspired local engineers in Ladakh, India to try
conserving winter water supply as ice structures. Despite the two regions being in different countries, the
geographic and cultural proximity of the communities paved the way for the further development of these
practices in Ladakh.
