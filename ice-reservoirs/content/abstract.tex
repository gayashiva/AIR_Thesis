% !TEX root = ../my-thesis.tex
%
\pdfbookmark[0]{Abstract}{Abstract}
\addchap*{Abstract}
\label{sec:abstract}

Irrigated agriculture is crucial for the livelihood security of mountain communities. Using meltwater from
glaciers, snow and permafrost, mountain dwellers have developed sophisticated techniques to cope with recurrent
water scarcity caused by glacial retreat, glacier thining, and seasonal snow-cover dynamics. Artificial ice
reservoirs (AIRs) are a key example of such a water storage technology. Worldwide, more than 500 mountain
farmers build these ice structures. These seasonal ice reservoirs increase meltwater availability during the
critical period of water scarcity in spring. To assesss the role of AIRs within the water resource management of
mountain villages under a changing climate, they need to be represented in integrated modelling frameworks like
glacier models. To reduce their water losses, their construction strategies need to be made sensitive to the
location's weather conditions and water availability. This thesis aims both to implement the volume evolution of
AIRs in a glacier mass and energy balance model framework and develop construction strategies that can enhance
their size and survival duration while increasing their water-use efficiency. 

To start, we estimate the differing contribution of AIR surface processes built in Guttannen, Switzerland and
Ladakh, India. These two locations exhibit different meteorological patterns owing to their significant
latitude, longitude and altitude differences. Using an AIR-specific mass and energy balance model forced with
meteorological, fountain discharge and ice volume datasets, surface processes are quantified and compared across
the two locations. The results reveal that the sublimation process is driving the ice volume differences, and
fountain operation of both the AIRs utilized less than one-fifth of the water supply provided. This case study
therefore highlights the importance of colder, drier climates and fountain water supply management when building
AIRs with higher volumes and lower water losses.  

Then, we zoom in to the local scale to provide the first estimate of the water loss reduction achieved due to
fountain scheduling strategies. Fountain scheduling was realized through an automation system computing
recommended discharge rates using real-time weather input and location metadata. The automation software was
developed by extending the AIR model to function as a discharge scheduler software providing the recommended discharge
rates. Simulations converting unscheduled fountains to scheduled fountains improved the water use efficiency of
several AIRs more than three fold. Fountain operation using scheduling strategies produced similar ice volumes
while consuming one-tenth of the water the unscheduled fountain used.  Overall, these results show that
automated fountain water supply management can both increase the water use efficiency of AIRs and reduce their
maintenance without compromising on their meltwater production.

Overall, this thesis advances the current understanding on the volume evolution of AIRs under different
climates. It provides tools to quantify the storage potential of these ice structures worldwide and practical
strategies to improve their efficacy. Future work may build on this research by fully integrating fountain
scheduling and climate change scenarios to investigate potential adaptation strategies for water-stressed
mountain communities.

% \vspace*{20mm}

% {\usekomafont{chapter}Abstract (different language)}
% \label{sec:abstract-diff}

% \blindtext
