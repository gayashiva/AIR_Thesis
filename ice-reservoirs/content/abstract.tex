% !TEX root = ../my-thesis.tex
%
\pdfbookmark[0]{Abstract}{Abstract}
\addchap{Abstract}
\label{sec:abstract}

Irrigated agriculture is crucial for the livelihood security of mountain communities. Using meltwater from
glaciers, snow and permafrost, mountain dwellers have developed sophisticated techniques to cope with recurrent
water scarcity caused by glacial retreat, glacier thinning, and seasonal snow-cover dynamics. Artificial ice
reservoirs (AIRs) are a key example of community based water management. Worldwide, more than 30 mountain
villages build these ice structures. These seasonal ice reservoirs increase meltwater availability during the
critical period of water scarcity in spring. To assess the role of AIRs within the water resource management of
mountain villages under a changing climate, they need to be represented in integrated modelling frameworks. To
achieve efficient water storage, their design needs to take into account the location's weather conditions and
water availability. This thesis aims to examine the volume evolution of AIRs within the framework of a mass and
energy balance model, as well as developing construction strategies that can enhance their size and duration
while increasing their water-use efficiency. 

To start, we estimate the differing contribution of AIR surface processes built in Guttannen, Switzerland and
Ladakh, India. These two locations exhibit different meteorological patterns due to their significant latitude,
longitude and altitude differences. Using an AIR-specific mass and energy balance model which keeps into account
meteorological factors, fountain discharge and ice volume changes, surface processes are quantified and compared
across the two locations. The results reveal that higher sublimation process enhanced the Indian AIRs freezing
rate, and poor fountain operation of both the AIRs resulted in wastage of four-fifth of the water supply
provided. These results therefore highlight the relevance of colder, drier climates and fountain water supply
management in optimizing  AIR construction.  

Then, we zoom into the Swiss location to provide the first estimate of the water loss reduction achieved due to
fountain scheduling strategies. Fountain scheduling was realized through a control valve that was automated with
optimal discharge rates computed using real-time weather input and location metadata. Simulations converting
unscheduled fountains to scheduled fountains showed a more than threefold improvement in the water use
efficiency of several AIRs. Fountain operation using scheduling strategies produced similar ice volumes while
consuming one-tenth of the water compared to their unscheduled counterparts.  Overall, these results show that
automated fountain water supply management can both increase the water use efficiency of AIRs and reduce their
maintenance needs without compromising on their meltwater production.

This thesis advances the current understanding on the volume evolution of AIRs under different climates. It
provides tools to quantify the storage potential of these ice structures worldwide and practical strategies to
improve their efficacy. This study provides the scientific evidence needed to upscale this indigenous water
storage technology, thus challenging the preconception that local water management traditions are outdated and
supporting the uptake of nature-based solutions for water security. Future work may build on this research by
fully integrating climate change scenarios to investigate the potential hydrological contributions of ice
harvesting technologies for water-stressed mountain catchments.
