\chapter{Science of ice reservoirs}
\label{chap:science}

\cleanchapterquote{I could do with some scientific help from specialists. I am trying to collect data on how and
	where glaciers form best so that I can improve on them and people can use the technique elsewhere.
}{Chewang Norphel}{(Padmashree awardee, Inventor of ice terraces)}

Given the importance of \ac{AIRs} in different mountainous regions, it is key to adequately understand their
functioning and to quantify their water storage at global scale. Addressing these research questions calls for
the use of holistic modelling frameworks in which all relevant processes are represented, including components
of the terrestrial hydrological cycle, human water management, atmospheric processes, and climate change
drivers, in a globally integrated way. In such coupled frameworks, interactions and feedback between ice surface
and climate are directly modelled and represented under different scenarios and for different time periods,
allowing the investigation of both the physical mechanisms and the spatial and temporal extents of water
storage. To date, glacial models have provided this type of integrated framework but without accurate
representations of \ac{AIRs}. However, with some modifications, glacial models can become ideal tools to
estimate the meltwater quantities of \ac{AIRs} for the historical and present-day periods and for scenarios of
future climate change. Such model-based assessments can aid future mitigation and adaptation strategies linked
to \ac{AIR} construction and operation to assess future water scarcity and ensure future water availability in a
changing climate.

To achieve this, three physically based models of increasing complexity are developed in the present thesis to
estimate AIR volume evolution. The first one, referred to as Oerlemans model (paper III), integrates a simple
shape evolution module to demonstrate that \ac{AIR} volume estimations are possible using equations used for
modelling glacial surfaces. The second one, referred to as \ac{AIR} model (paper I), extends this approach to
calibrate the model parameters and validate the volume estimations it provided. The third one, referred to as
COSISTUPA, reduces model calibration overhead while providing a user-friendly framework to update model
parametrizations and apply the model to new construction sites. 

In this chapter, the datasets obtained across measurement campaigns in two different mountain regions (the Alps
and the Himalayas) are presented. Description of the shape evolution and mass and energy balance modules used to
assess quantity of ice, meltwater, sublimation, and spillwater are also provided herein. Finally, features and
shortcomings of each model are compared.

\section{Study sites and data}

The study sites in the Alps and the Himalayas were chosen for two reasons. First, they enable a comparative
study between locations with considerable differences in meteorological characteristics. Second, these locations
do not present significant logistical hurdles, since both selected sites have dedicated teams for construction
and measurement campaigns.

The study period starts when the fountain is first switched on (start date) and ends when the respective
\ac{AIR} either melts or brakes into several ice blocks (expiry date). Each \ac{AIR} dataset was abbreviated
based on the construction strategy used, prefix of the country code, and suffix of the year of its expiry date
(Table \ref{tab:AIRs}). The construction strategies are distinguished based on whether they use fountain
scheduling strategies or not to regulate water supply. Fountain scheduling was set with a control valve that was
automated with optimal discharge rates computed using real-time meteorological input and location metadata.
Those using fountain scheduling were code named "automated", whereas the rest were code named "manual". All
except one construction campaign used manual construction strategies. Therefore, manual \ac{AIRs} are referred
to without explicitly specifying their construction strategy.

In total, 23 \ac{AIRs} were studied in these two regions across four winters. However, complete meteorological,
water, and volume measurements were available for only four \ac{AIRs} located in Guttannen, Switzerland and one
in Gangles, India. Therefore, only these \ac{AIR} datasets are used in the following analysis. The rest \ac{AIR}
datasets are described in the Appendix table \ref{tab:Ladakh_AIRs} and are used in later chapters for
qualitative analysis.

\begin{figure}[htb]
	\centering
	\includegraphics[width=12 cm]{figs/2AIRs.jpg}
	\caption{The Swiss and Indian \ac{AIRs} were 5 $m$ and 13 $m$ tall on January 9 and March 3, 2021,
    respectively. Photos: Daniel Bürki (left) and Thinles Norboo (right).}
	\label{fig:2AIRs}
\end{figure}

\subsection{Swiss site}

The Guttannen site (46.66 $\degree$N, 8.29 $\degree$E) is situated in the Berne region, Switzerland and presents
an altitude of 1047 $m$ a.s.l. In the winter (Oct–Apr), mean daily minimum and maximum air temperatures vary
between $-13$ and 15 $\degree C$. Clear skies are rare, averaging around 7 days during winter. Daily winter
precipitation can sometimes be as high as 100 $mm$. These values are based on 30 years of hourly historical
meteorological data series \citep{meteoblueClimateGuttannen2021}. Several \ac{AIRs} were constructed by the
Guttannen Bewegt Association, the University of Fribourg, and the Lucerne University of Applied Sciences and
Arts during the winters of 2020–22.

\subsection{Indian site}

The Gangles site (34.22 $\degree$N, 77.61 $\degree$E) is located around 20 km north of Leh city in the Ladakh
region, lying at 4025 $m$ a.s.l.. The mean annual temperature is $5.6 \, \degree C$, and the thermal range is
characterized by high seasonal variation. During January, the coldest month, the mean temperature drops to $-7.2
\, \degree C$. During August, the warmest month, the mean temperature rises to $17.5 \, \degree C$
\citep{nusserIrrigationDevelopmentUpper2012}. Because of the rain shadow effect of the Himalayan range, the mean
annual precipitation in Leh totals less than 100 $mm$, and there is high interannual variability. While the
average summer rainfall between July and September reaches 37.5 $mm$, the average winter precipitation between
January and March amounts to 27.3 $ mm$ and falls almost entirely as snow. \ac{AIRs} were constructed here by
the Himalayan Institute of Alternatives, Ladakh during the winter of 2020/21.

\subsection{Meteorological data}

Air temperature, relative humidity, wind speed, pressure, longwave, and global shortwave radiation are required
as model input. The resulting dataset highlights the difference in meteorological influences driving ice volume
evolution in the two study sites (Table \ref{tab:Observations}).

\begin{table}
	\centering
	\caption{Summary of the meteorological observations for \ac{AIRs} built during the respective study period.
		The meteorological measurements are shown using their mean ($\mu$) and standard deviation ($\sigma$) during the study
		period as $\mu \pm \sigma$. }

	\label{tab:Observations}
	\begin{tabular}{|lllll|}
		\hline
		\textbf{Name}        & \textbf{Symbol} & \textbf{IN21} & \textbf{CH21} & \textbf{Units} \\ \hline
		Air temperature      & $T_a    $       & $0 \pm 7$     & $2 \pm 6$     & $\degree C$    \\
		Relative humidity    & $RH     $       & $35 \pm 20$   & $79 \pm 18$   & \%             \\
		Wind speed           & $v_a        $   & $3 \pm 1$     & $2 \pm 2$     & $m/s$          \\
		Global shortwave     & $SW_{global} $  & $246 \pm 333$ & $138 \pm 243$  & $W\,m^{-2}$    \\
		Precipitation        & $ppt        $   & $0 \pm 0$     & $139 \pm 457$ & $mm$           \\
		Pressure             & $p_a         $  & $623 \pm 3$   & $794 \pm 9$   & $hPa$          \\\hline
	\end{tabular}
\end{table}

\subsection{Fountain observations}

A fountain consists of a pipeline and a nozzle. The pipeline has three attributes, namely discharge rate
($Q$), height ($h_F$), and water temperature ($T_F$). "Discharge rate" represents the discharge rate of the water in
the fountain pipeline. "Height" denotes the height of the fountain pipeline installed. "Fountain water temperature"
is the temperature of water droplets produced by the fountain.

\begin{figure}
	\centering
	\includegraphics[width=\textwidth/2]{figs/CH20_sprayrad.jpg}
	\caption{Spray radius of the CH20 \ac{AIR}. }
	\label{fig:CH20_rad}
\end{figure}

The fountain nozzle has two main characteristics, namely the aperture diameter ($dia_{F}$) and pressure loss
($P_{F}$). "Pressure loss" denotes the loss of water head due to the fountain nozzle. Additionally,
the observed ice radius formed from the fountain water droplets is denoted as spray radius ($r_F$) (Fig.
\ref{fig:CH20_rad}).


\subsection{Drone flights}

\begin{table}
	\centering
	\caption{List of all \ac{AIRs} studied. The study period starts when the fountain is first switched on
		(denoted as Start Date) and ends when the respective \ac{AIR} either melts or brakes into several ice blocks
		(denoted as Expiry Date). }
	\label{tab:AIRs}
	\begin{tabular}{|lllll|}
		\hline
		\textbf{Name}    & \textbf{Start Date} & \textbf{Expiry Date} & \textbf{No. of flights} & \textbf{Spray
		radius}                                                                                                 \\ \hline
		Manual CH20 & Jan 3 2020          & Apr 6 2020           & 2                       & 7.7 $m$       \\
		Manual CH21 & Nov 22 2020         & May 10 2021          & 8                       & 6.9 $m$       \\
		Manual IN21 & Jan 18 2021         & June 20 2021         & 6                       & 10.2 $m$      \\
		Manual CH22 & Dec 8 2021          & April 12 2022        & 8                       & 4.1 $m$       \\
		Automated CH22   & Dec 8 2021          & April 12 2022        & 6                       & 4.8 $m$       \\ \hline
	\end{tabular}
\end{table}

Several photogrammetric surveys were conducted for each \ac{AIR}. Details of these surveys and the
methodology used to produce the corresponding outputs are explained in paper I. The
\ac{DEMs} generated from the obtained imagery were analyzed to document the ice radius, the surface area, and the
volume of the ice structures. Ice radius measurements of drone flights which observed an increase in either \ac{AIR}
circumference or volume were averaged to determine the fountain's spray radius. The number of drone surveys
conducted for each \Ac{AIR} and the corresponding spray radius observed are presented in Table \ref{tab:AIRs}.

\subsection{Ground penetrating radar measurements}

\ac{GPR} surveys were conducted on four ice stupas in March, 2020. The surveys were conducted using a portable
\ac{GPR}, with a frequency of 400 $MHz$, which was towed with the help of ropes over profiles marked on the ice
stupa (Fig. \ref{fig:gpr_survey}). The profiles generated from these surveys were analyzed to document the
volume of the ice structures. 

\begin{figure}
  \centering
	\includegraphics[width=8 cm]{figs/gpr_survey}
  \caption{Portable \ac{GPR}s carried across the surface of \ac{AIRs} to produce the associated datasets.
    (Photo: Shijay Projects).}
	\label{fig:gpr_survey}
\end{figure}

The basic principle of a pulsed \ac{GPR} system is to send an electromagnetic signal into the ground and to
record the signal reflections as a function of their two-way travel time. Partial reflections of the
electromagnetic wave recorded as \ac{IRH} occur at vertical discontinuities in the dielectric material. From
polar studies, \ac{IRH} are known to coincide with variations in density and liquid water content
\citep{forster2014extensive}. Further details of these surveys and the methodology used to produce the
corresponding outputs are explained in \citet{balasubramanian_suryanarayanan_2022_7056646}.



\section{Model modules}
\label{sec:modules}

A bulk energy and mass balance model is used to calculate the amounts of ice, meltwater, water vapor, and
spillwater of the \ac{AIR}. In each hourly time step, the model uses the \ac{AIR} surface area, energy, and mass balance
calculations to estimate its ice volume, surface temperature, and spillwater, as shown in Fig. \ref{fig:schema}.

\begin{figure}
	\begin{center}
		\includegraphics[width=10 cm]{figs/model_schematic.jpg}
	\end{center}
	\caption{Model schematic showing the workflow used in the model at every time step. }
	\label{fig:schema}
\end{figure}

\subsection{Shape evolution module} \label{sec:shape}

The model assumes the \ac{AIR} shape to be a cone and assigns the following shape attributes:

\begin{subequations}
	\begin{align}
		\label{eq:A}
		A_{cone}^i & = \pi \cdot r_{cone}^i \cdot \sqrt{{(r_{cone}^i)}^2 + {(h_{cone}^i})^ 2} \\
		\label{eq:V}
		V_{cone}^i & = \pi/3 \cdot {(r_{cone}^i)}^2 \cdot h_{cone}^i                          \\
		\label{eq:thickness}
		j_{cone}^i & =\frac{\Delta M_{ice}^i}{\rho_{water}* A_{cone}^i}
	\end{align}
\end{subequations}

where $i$ denotes the model time step; $r_{cone}^i$ is the radius; $h_{cone}^i$ is the height; $A_{cone}^i$ is
the surface area; $V_{cone}^i$ is the volume; and $j_{cone}^i$ is the \ac{AIR} surface normal thickness change as shown
in Fig. \ref{fig:shape}. $M_{ice}^i$ is the mass of the \ac{AIR}, and $\Delta M_{ice}^i = M_{ice}^{i-1} -
	M_{ice}^{i-2}$. Henceforth, the equations used display the model time step superscript $i$ only if this is different
from the current time step.

AIR density can be defined as:

\begin{equation}
	\rho_{cone} = \frac{M_{F} + M_{dep} + M_{ppt}}{(M_{F} + M_{dep})/\rho_{ice} + M_{ppt}/\rho_{snow}}
\end{equation}

where $M_F$ is the cumulative mass of the fountain discharge; $M_{ppt}$ is the cumulative precipitation;
$M_{dep}$ is the cumulative accumulation through water vapor deposition; $\rho_{ice}$ is the ice density (917
$kg\,m^{-3}$); and $\rho_{snow}$ is the density of wet snow (300 $kg\,m^{-3}$) taken from
\cite{cuffeyPhysicsGlaciers2010}.

AIR volume can also be expressed as:

\begin{equation} V_{cone} =\frac{M_{ice}} {\rho_{cone}} \label{eq:V1} \end{equation}

The initial radius of the \ac{AIR} is assumed to be $r_F$. The initial height $h_0$ depends on the dome volume
$V_{dome}$ used to construct the \ac{AIR} as follows:

\begin{equation}
	h_{0} =  \Delta x + \frac{3 \cdot V_{dome}}{\pi \cdot (r_F)^2 }
	\label{eq:h0}
\end{equation}

where $\Delta x$ is the surface layer thickness (defined in Section \ref{sec:energy}).

During the subsequent time steps, the dimensions of the \ac{AIR} evolve assuming a uniform thickness change ($j_{cone}$)
across its surface area with an invariant slope $s_{cone} = \frac{h_{cone}}{r_{cone}}$ .  During these time
steps, the volume is parameterized using Eqn. \ref{eq:V} as:

\begin{equation} 
  V_{cone} = \frac{\pi \cdot {(r_{cone})}^3 \cdot s_{cone}}{3} 
\label{eq:V2} 
\end{equation}

The ice stupa boundary is defined through its spray radius, i.e., ice formation is assumed negligible when
$r_{cone} > r_{F}$. Combining Eqns. \ref{eq:V},  \ref{eq:V1}, \ref{eq:h0}, and \ref{eq:V2}, the geometric
evolution of the ice stupa at each time step $i$ can be determined by considering the following rules:

\begin{equation} (r_{cone},\, h_{cone}) = \left\{ \begin{array}{ll} (r_F ,\, h_0)                                                                          & \textit{ if } i=0 \\
             (r_{cone}^{i-1},\, \frac{3 \cdot M_{ice}}{\pi \cdot \rho_{ice} \cdot {(r_{cone}^{i-1})}^2}) & \textit{ if }
             r_{cone}^{i-1} \geq r_{F} \textit{ and } \Delta M_{ice} > 0                                                     \\ (\frac{3 \cdot M_{ice}}{\pi \cdot \rho_{ice} \cdot s_{cone}})^{1/3} \cdot (1,\,  s_{cone}) &
             otherwise\end{array} \right.  \label{eq:A2} \end{equation}

\subsection{Energy balance module} \label{sec:energy}

\begin{figure}
	\begin{center}
		\includegraphics[width=10 cm]{figs/AIR_schematic.jpeg}
	\end{center}
	\caption{Shape variables of the \ac{AIR}. $r_{cone}$ is the radius; $h_{cone}$ is the height; $j_{cone}$ is the
		thickness change; and $s_{cone}$ is the slope of the ice cone. $r_F$ is the spray radius of the fountain.}
	\label{fig:shape}
\end{figure}

The energy balance at the surface of an \ac{AIR} is approximated by a one-dimensional description of energy fluxes
into and out of a (thin) layer with thickness $\Delta x$:

\begin{equation}
	\rho_{cone} \cdot c_{ice} \cdot \frac{\Delta T}{\Delta t} \cdot \Delta x = q_{SW} + q_{LW} + q_{L} + q_{S} + q_{F}+ q_{R} + q_{G}
	\label{eqn:EB}
\end{equation}

Upward and downward fluxes relative to the ice surface are positive and negative, respectively. The first term
of the equation is the energy change of the surface layer, which can be translated into a phase change energy should phase
changes occur. $q_{SW}$ is the net shortwave radiation; $q_{LW}$ is the net longwave radiation; $q_{L}$ and
$q_{S}$ are the turbulent latent and sensible heat fluxes, respectively; $q_{F}$ and $q_{R}$ represent the heat exchange of
the fountain water droplets and rain droplets with the \ac{AIR} ice surface, respectively; and $q_{G}$ represents bulk
heat flux between the \ac{AIR} surface and its interior.

The energy flux acts upon the \ac{AIR} surface layer, which has an upper and lower boundary defined by the atmosphere
and the ice body of the \ac{AIR}, respectively. Here, the surface temperature $T_{ice}$ is defined to be the modelled
average temperature of the ice stupa surface layer.

\subsubsection{Net shortwave radiation \texorpdfstring{$q_{SW}$}{Lg}}

The net shortwave radiation $q_{SW}$ is computed as follows:

\begin{equation} q_{SW} = (1- \alpha)\cdot (SW_{direct} \cdot f_{cone} + SW_{diffuse}) \label{eqn:SW} \end{equation}

where $SW_{direct}$ and $SW_{diffuse}$ are the direct and diffuse shortwave radiation, respectively; $\alpha$ is the
modelled albedo; and $f_{cone}$ is the area fraction of the ice structure exposed to direct shortwave
radiation.

The albedo varies depending on the water source that formed the current \ac{AIR} surface layer. During the fountain
runtime, the albedo assumes a constant value corresponding to ice albedo. However, after the fountain is
switched off, the albedo can reset to snow albedo during snowfall events and then decay back to ice albedo. The scheme described in \cite{oerlemansYearRecordGlobal1998} is used to model this process. The scheme records the
decay of albedo with time after fresh snow is deposited on the surface. $\delta t$ records the number of time
steps after the last snowfall event. After snowfall, albedo changes over a time step, $\delta t$ , as:

\begin{equation} \alpha=\alpha_{ice}+(\alpha_{snow}-\alpha_{ice}) \cdot e^{(-\delta t)/\tau} \label{eqn:alb}
\end{equation}

where $\alpha_{ice}$ is the bare ice albedo value (0.25); $\alpha_{snow}$ is the fresh snow albedo value (0.85);
and $\tau$ is a decay rate (16 days), which determines how fast the albedo of the ageing snow recedes back to
ice albedo. Discharge events decrease the decay rate by a factor of $\alpha_{ice}/\alpha_{snow}$.

The solar area fraction $f_{cone}$ of the ice structure exposed to direct shortwave radiation depends on the shape
considered. Using the solar elevation angle $\theta_{sun}$, the solar beam can be considered to present a vertical
component, impinging on the horizontal surface (semicircular base of the \ac{AIR}), and a horizontal component,
impinging on the vertical cross section (a triangle). The solar elevation angle $\theta_{sun}$ used is modelled
using the parametrization proposed by \cite{woolfComputationSolarElevation1968}. Accordingly, $f_{cone}$ is determined as follows:

\begin{equation}
	\begin{split}
		f_{cone}& =\frac{(0.5 \cdot r_{cone} \cdot h_{cone}) \cdot cos \theta_{sun} +(\pi \cdot
		{(r_{cone})}^2/2) \cdot sin \theta_{sun} }{\pi \cdot r_{cone} \cdot ({(r_{cone})}^2+{(h_{cone})}^2)^{1/2}}\\
	\end{split}
	\label{eqn:f_{cone} }
\end{equation}

The diffuse shortwave radiation is assumed to impact the conical \ac{AIR} surface uniformly.

\subsubsection{Net longwave radiation \texorpdfstring{$q_{LW}$}{Lg}} \label{sec:LW}

The net longwave radiation $q_{LW}$ is determined as follows:

\begin{equation}
	q_{LW}= LW_{in}-\sigma \cdot \epsilon_{ice} \cdot {(T_{ice}+ 273.15)}^4
	\label{eqn:LW}
\end{equation}

where $T_{ice}$ is the modelled surface temperature given in [$\degree C$];
$\sigma=5.67\cdot10^{-8}\,Jm^{-2}s^{-1}K^{-4}$ is the Stefan–Boltzmann constant; $LW_{in}$ denotes the incoming
longwave radiation; and $\epsilon_{ice}$ is the corresponding emissivity value for the ice stupa surface (0.97).

The incoming longwave radiation $LW_{in}$ for the Indian site, for which no direct measurements are available, is
determined as follows:

\begin{equation}
	LW_{in}=\sigma \cdot \epsilon_a \cdot {(T_a+ 273.15)}^4
	\label{eqn:LWin}
\end{equation}

where $T_a$ represents the measured air temperature and $\epsilon_a$ denotes the atmospheric emissivity. The atmospheric emissivity $\epsilon_a$ is approximated using the equation suggested by \cite{brutsaertEvaporationAtmosphereTheory1982},
considering air temperature and vapor pressure (Eqn.  \ref{eqn:atm_e}). The vapor pressure of air over water and
ice is obtained using Eqn. \ref{eqn:vp}. The expression defined in \cite{brutsaertDerivableFormulaLongwave1975} for clear skies
(first term in equation \ref{eqn:atm_e}) is extended with the correction for cloudy skies after
\cite{brutsaertEvaporationAtmosphereTheory1982} as follows:

\begin{equation}
	\epsilon_a=1.24 \cdot (\frac{p_{v,w}}{(T_a+273.15)})^{1/7}\cdot(1+0.22\cdot{cld}^2) \label{eqn:atm_e}
\end{equation}

with a cloudiness index $cld$, ranging from 0 for clear skies to 1 for complete overcast skies. For the Indian
site, cloudiness is assumed to be negligible.

\subsubsection{Turbulent fluxes} \label{sec:Qs}

The turbulent sensible $q_{S}$ and latent heat $q_{L}$ fluxes are computed with the following expressions
proposed by \cite{garrattAtmosphericBoundaryLayer1992}:

\begin{equation}
	q_{S}=\mu_{cone}\cdot c_{a} \cdot \rho_{a} \cdot p_{a}/p_{0,a} \cdot \frac{\kappa^2 \cdot v_a \cdot
		(T_a-T_{ice})}{{(\ln{\frac{h_{AWS}}{z_{0}}})}^2}
	\label{eqn:qs}
\end{equation}

\begin{equation}
	q_{L}=\mu_{cone}\cdot 0.623 \cdot L_s \cdot \rho_{a}/p_{0,a} \cdot \frac{\kappa^2 \cdot
	v_a(p_{v,w}-p_{v,ice})}{{(\ln{\frac{h_{AWS}}{z_{0}}})}^2}
\end{equation}

where $h_{AWS}$ is the measurement height above the ground surface of the \ac{AWS} (around $2\,m$ for all sites);
$v_a$ is the wind speed in [$m\,s^{-1}$]; $c_a$ is the specific heat of air at constant pressure (1010 J
$kg^{-1} K^{-1}$); $\rho_{a}$ is the air density at standard sea level (1.29 $kg m^{-3}$); $p_{0,a}$ is the air
pressure at standard sea level (1013 $hPa$); $p_{a}$ is the measured air pressure; $\kappa$ is the von Karman
constant (0.4); $z_{0}$ is the surface roughness (3 $mm$); and $L_s$ is the heat of sublimation (2848
$kJ\,kg^{-1}$). The vapor pressure of air with respect to water ($p_{v,w}$) and with respect to ice
($p_{v,ice}$) is obtained using the formulation given in \cite{huangSimpleAccurateFormula2018}:

\begin{equation}
	\begin{split}
		p_{v,w}&=e^{\frac{(34.494 - \frac{4924.99}{T_{a} + 237.1})}{(T_a + 105)^{1.57} \cdot 100}} \cdot \frac{RH}{100} \\
		p_{v,ice}&=e^{\frac{(43.494 - \frac{6545.89}{T_{ice} + 278})}{(T_{ice} + 868)^{2} \cdot 100}} \\
	\end{split} \label{eqn:vp}
\end{equation}

The dimensionless parameter $\mu_{cone}$ is an exposure parameter that deals with the fact that an \ac{AIR} presents a rough
appearance and forms an obstacle to the wind regime. This factor accounts for the larger turbulent fluxes due to
the roughness of the surface \citep{oerlemansBriefCommunicationGrowth2021} and is a function of \ac{AIR} slope
as follows:

\begin{equation}
	\mu_{cone} = 1 + \frac{s_{cone}}{2}
	\label{eqn:mu}
\end{equation}

The use of wind measurements from the horizontal plane at the \ac{AWS} can represent a possible source of error,
as these measurements might be different from those on a slope. However, this assumption is retained due to the absence of detailed datasets from the \ac{AIR} surface.

\subsubsection{Fountain discharge heat flux \texorpdfstring{$q_{F}$}{Lg} } \label{sec:heatflux}

Fountain water temperature, $T_F$, is assumed to cool to 0 $\degree C$ after contact with the ice surface.
$T_F$ is equal to the measured source water temperature, but during time periods when the ambient temperature is
subzero, $T_F$ is assumed to be 0 $\degree C$. Thus, the heat flux caused by this process is:

\begin{equation}
	q_{F} = \left\{ \begin{array}{ll}
		\frac{ \Delta M_F \cdot c_{water} \cdot T_F}{\Delta t \cdot A_{cone}} & \textit{ if } T_{temp} > 0 \\
		0                                                                     & \textit{ otherwise}
	\end{array} \right.
\end{equation}

with $c_{water}$ as the specific heat of water (4186 J $kg^{-1} K^{-1}$).

\subsubsection{Rain heat flux \texorpdfstring{$q_{R}$}{Lg} }

The influence of rain events on the albedo and on the ice stupa's energy balance is assumed to be similar to that of discharge
events. However, the water temperature of a rain event is assumed to be equal to the air temperature. Accordingly,
the heat flux generated due to a rain event is determined:

\begin{equation}
	q_{R} = \frac{ \Delta M_{ppt} \cdot c_{water} \cdot T_a}{\Delta t \cdot A_{cone}}
	\label{eqn:qR}
\end{equation}

\subsubsection{Bulk heat flux \texorpdfstring{$q_{G}$}{Lg}} \label{sec:Bulkflux}

The bulk heat flux $q_{G}$ corresponds to the ground heat flux in normal soils and is caused by the
temperature gradient between the surface layer ($T_{ice}$) and the ice body ($T_{bulk}$). It is expressed by
using the heat conduction equation as follows:

\begin{equation} q_{G} = k_{ice} \cdot (T_{bulk}-T_{ice}^{i-1})/l_{cone} \label{eqn:qG}    \end{equation}

where $k_{ice}$ is the thermal conductivity of ice (2.123 $W\, m^{-1}\,K^{-1}$); $T_{bulk}$ is the mean
temperature of the ice body within the ice stupa; and $l_{cone}$ is the average distance of any point on the
surface to any other point in the ice body. $T_{bulk}$ is initialized as 0 $\degree C$ and later determined from
Eqn. \ref{eqn:qG} as follows:

\begin{equation} T_{bulk}^{i+1} = T_{bulk} - (q_{G} \cdot A \cdot \Delta t)/(M_{ice} \cdot c_{ice}) \end{equation}

Since \ac{AIRs} typically present conical shapes with $r_{cone} > h_{cone}$, the center of mass of the cone body is
assumed to be near the base of the fountain. Thus, the distance of every point on the \ac{AIR} surface layer from the
cone body's center of mass is between $h_{cone}$ and $r_{cone}$. Therefore, $q_{G}$ is calculated assuming
$l_{cone} = (r_{cone} + h_{cone})/2$.

\subsubsection{Phase changes} \label{sec:phase}

This section explains the numerical procedures to model phase changes at the surface layer. Letting
$T_{temp}$ be the calculated surface temperature, Eqn. \ref{eqn:EB} can be rewritten as:

$$q_{total} =\rho_{ice} \cdot c_{ice} \cdot \frac{(T_{temp}-T_{ice})}{\Delta t} \cdot \Delta x$$

where $q_{total}$ represents the total energy available to be redistributed. Even if the numerical heat transfer
solution produces temperatures $T_{temp}>0\, \degree C$, for instance from intense shortwave radiation, the ice
temperature must remain at $T_{temp} = 0\, \degree C$. The "excess" energy is used to drive the melting
process. Moreover, the energy input is used to melt the surface ice layer, not to raise the surface
temperature to some unphysical value. Similarly, for freezing to occur, three conditions are required. First,
fountain water must be present ($\Delta M_{F} > 0 $); second, the calculated temperature of the ice, $T_{temp}$,
must be below $0\, \degree C$. However, these two conditions are not sufficient, as the latent heat turbulent fluxes
can only contribute to temperature fluctuations. Therefore, an additional condition, namely $(q_{total}-q_{L})
	< 0$, is required. Depending on the above conditions, the total energy $q_{total}$ can be redistributed
for the melting ($q_{melt}$), freezing ($q_{freeze}$), and surface temperature change ($q_{T}$) processes as
follows:

\begin{equation}
	q_{total} = \left\{ \begin{array}{ll}
		q_{freeze} + q_{T} & \textit{ if } \Delta M_{F} > 0 \textit{ and } T_{temp} < 0 \textit{ and }(q_{total}-q_{L}) < 0 \\
		q_{melt} + q_{T}   & \textit{ otherwise}
	\end{array} \right.
\end{equation}

Henceforth, time steps when total energy is redistributed to the freezing energy are called freezing
events and, the rest of the time, steps are called melting events.


During a freezing event, the \ac{AIR} surface is assumed to warm to $0 \degree C$. The available energy
$(q_{total}-q_{L})$ is further increased due to this change in surface temperature represented by the energy
flux:

$$q_{0} = \frac{\rho_{ice} \cdot \Delta x \cdot c_{ice} \cdot T_{ice}^{i-1}}{\Delta t}$$

The available fountain discharge ($\Delta M_{F}$) may not be sufficient to utilize all the freezing energy. At such times,
the additional freezing energy further cools down the surface temperature. Accordingly, the surface energy flux
distribution during a freezing event can be represented as:

\begin{equation}
	(q_{freeze}, q_{T}) = \left\{ \begin{array}{ll}
		(\frac{\Delta M_{F} \cdot L_f
		}{A_{cone} \cdot \Delta t}
		, q_{total}+\frac{\Delta M_{F} \cdot L_f
		}{A_{cone} \cdot \Delta t})          & \textit{ if  } \Delta M_{F} \textit{ insufficient } \\
		(q_{total}-q_{L}+q_{0}, q_{L}-q_{0}) & \textit{ otherwise }                                \\
	\end{array} \right.
\end{equation}

If $T_{temp} > 0 \degree C$, then energy is reallocated from $q_{T}$ to $q_{melt}$ to maintain surface
temperature at melting point. The total energy flux distribution during a melting event can be represented as:

\begin{equation}
	(q_{melt}, q_{T}) = \left\{ \begin{array}{ll}
		(0, q_{total})
		                                                                                                                                                               & \textit{ if } T_{temp} \leq 0 \\
		(\frac{T_{temp} \cdot \rho_{ice} \cdot c_{ice} \cdot \Delta x}{\Delta t}, q_{total}-\frac{T_{temp} \cdot \rho_{ice} \cdot c_{ice} \cdot \Delta x}{\Delta t}  ) & \textit{ if } T_{temp} > 0
	\end{array} \right.
\end{equation}

\subsection{Mass balance module}

The mass balance equation for an \ac{AIR} is represented as:

\begin{equation}
	\frac{\Delta M_{F} + \Delta M_{ppt} + \Delta M_{dep}}{\Delta t} = \frac{\Delta M_{ice} +\Delta M_{water} +
		\Delta M_{sub} + \Delta M_{waste}}{\Delta t}  \\
	\label{eq:MB}
\end{equation}

where $M_{F}$ is the cumulative mass of the fountain discharge; $M_{ppt}$ is the cumulative precipitation;
$M_{dep}$ is the cumulative accumulation through water vapor deposition; $M_{ice}$ is the cumulative mass of
ice; $M_{water}$ is the cumulative mass of meltwater; $M_{sub}$ represents the cumulative water vapor loss by
sublimation; and $M_{waste}$ represents the fountain spillwater that did not interact with the \ac{AIR}. The
left hand side of the equation \ref{eq:MB} represents the rate of mass input, and the right hand side represents
the rate of mass output for an \ac{AIR}.

Precipitation input is calculated as shown in equation \ref{eq:ppt}, where $\rho_{w}$ is the density of water
(1000 $kg\,m^{-3}$), $\Delta ppt/ \Delta t$ is the measured precipitation rate in [$m\,s^{-1}$], and $T_{ppt}$
is the temperature threshold below which precipitation falls as snow. In the present work, snowfall events are
identified using $T_{ppt}$ as $1 \degree C$. Snow mass input is calculated by assuming a uniform deposition over
the entire circular footprint of the \ac{AIR}.

The latent heat flux is used to estimate either evaporation and condensation processes or sublimation and
deposition processes, as shown in equation \ref{eq:vap}. During the time steps at which the surface temperature
is below 0 $\degree C$, only sublimation and deposition can occur; but if the surface temperature reaches 0
$\degree C$, evaporation and condensation can also occur. As the differentiation between evaporation and
sublimation (and condensation and deposition) when the air temperature reaches 0 $\degree C$ is challenging,
negative (positive) latent heat fluxes are assumed to correspond only to sublimation (deposition), i.e., no
evaporation (condensation) is calculated.

Since every time step is categorized as a freezing or melting event, the melting/freezing rates and the
corresponding meltwater/ice quantities can be determined as shown in equations \ref{eq:m_freeze/melt},
\ref{eq:mwat}, and \ref{eq:mcone}. Having calculated all other mass components, the fountain spillwater generated
every time step can be calculated using Eqn. \ref{eq:MB}.

\begin{subequations}
	\begin{align}
		\frac{\Delta M_{F}}{\Delta t}                                      & = \left\{ \begin{array}{ll} \frac{60}{\rho_w \cdot \Delta t} \cdot d_F
			 & \textit{ if fountain is on} \\ 0 & \textit{ otherwise } \\\end{array} \right.                                             \\
		\label{eq:ppt}
		\frac{\Delta M_{ppt}}{\Delta t}                                    & = \left\{ \begin{array}{ll} \pi \cdot
			{(r_{cone})}^2 \cdot
			\rho_{w}\cdot \frac {\Delta ppt}{\Delta t} & \textit{ if } T_{a} < T_{ppt} \\ 0 & \textit{ if } T_{a} \geq T_{ppt} \\\end{array} \right.                                             \\
		\label{eq:vap}
		(\frac{\Delta M_{dep}}{\Delta t}, \frac{\Delta M_{sub}}{\Delta t}) & = \left\{ \begin{array}{ll} \frac{q_{L}
			\cdot A_{cone}}{L_s}\cdot (1,0)  & \textit{ if } q_{L} \geq 0 \\ \frac{q_{L}
			\cdot A_{cone}}{L_s}\cdot (0,-1) & \textit{ if } q_{L} < 0    \\\end{array} \right.                                             \\
		\label{eq:mwat}
		\frac{\Delta M_{water}}{\Delta t}                                  & = \frac{q_{melt} \cdot A_{cone} }{L_f}                                                   \\
		\label{eq:m_freeze/melt}
		\frac{\Delta M_{freeze/melt}}{\Delta t}                            & = \frac{q_{freeze/melt} \cdot A_{cone} }{L_f}                                            \\
		\label{eq:mcone}
		\frac{\Delta M_{ice}}{\Delta t}                                    & = \frac{q_{freeze}\cdot A_{cone} }{L_f} + \frac{\Delta M_{ppt}}{\Delta t} + \frac{\Delta
			M_{dep}}{\Delta t}- \frac{\Delta M_{sub}}{\Delta t}- \frac{\Delta M_{water}}{\Delta t}
	\end{align}
\end{subequations}

Considering \ac{AIRs} as water reservoirs, their net water loss can be defined as:

\begin{equation} \textit{Net water losses} = \frac{M_{waste}+M_{sub}}{(M_F+M_{ppt}+M_{dep})} \cdot 100 \end{equation}

\section{Overview of different models}
\label{sec:MIP}

The model modules described above are used in three different models with varying degrees of complexity, as
presented in Table \ref{tab:MIP}. The Oerlemans model is the simplest, implemented in just one page of code, whereas
the COSISTUPA model is the most complex, maintained by a community of modellers. The motivations involved
in developing each of these models are described below.

\begin{table}[ht]
	\centering
  \caption{Characteristics of the models used in the estimation of \ac{AIR} volumes. The models are referred to by
  the short name given in the first row. Full model documentation is provided in the original literature (marked in
  italics). }      

	\label{tab:MIP}
	\begin{tabular}{|llll|}
		\hline
		\textbf{Module}        & \textbf{Oerlemans} & \textbf{AIR} & \textbf{COSISTUPA}     \\ 
		\textit{Documentation} & \textit{Paper III} & \textit{Paper I} & \textit{\citet{sauterCOSIPYV1Opensource2020}} \\ 
		                       &                    &                  & \textit{and Section \ref{sec:Cosistupa}}     \\ \hline
		Shape evolution        & Constant radius     & Derived from  & Derived from        \\
                           & or slope            & energy balance & surface mass balance        \\\hline
		Density and temperature& None & 1-dimensional   & 2-dimensional   \\
		variation              &           &        & \\ \hline
	\end{tabular}
\end{table}

\subsection{Oerlemans model}

The objective of the Oerlemans model was to integrate a shape evolution module with glacial models to obtain
first-order estimates of growth and decay rates under various conditions. This model was designed to
qualitatively assess the effects of snow cover, different starting dates, differences between warm and cold
winters, etc. Ice stupa evolution was simulated with this model, using data from field measurements in the Oberengadin region, Switzerland. 

However, this model was not ideal to compare simulated ice stupa volumes with observed ones. This is
because the model considers the ice stupa to be a single unit with a surface temperature close to the melting
point. This assumption limits the quantification of surface processes which depend strongly on surface
temperature variations. Moreover, the model assumes unlimited water availability, which is often not the case
during \ac{AIR} construction. Also, the model uses a simplistic shape evolution module that ignores the dependence of
shape variables on fountain characteristics. These shortcomings made it necessary to extend the Oerlemans
model further.

\subsection{AIR model}

The \ac{AIR} model was designed to obtain better validation with measured ice stupa volumes than that the
Oerlemans model could offer. To do this, the model incorporates all mass and energy balance modules described in
Section \ref{sec:modules} and simulates \ac{AIR} evolution using data from field measurements in Gangles, India
and Guttannen, Switzerland. The model was calibrated and validated with ice volume and surface area observations
obtained via drone and \ac{GPR} surveys (Fig. \ref{fig:Cosistupa}). The freezing and melting rates for each
\ac{AIR} were calculated, and their corresponding magnitudes in terms of influence of the chosen location and
the fountain used were explained. The model was successful in reproducing the observed ice volume evolution with
a correlation greater than 0.96 and an \ac{RMSE} less than 18\% of the maximum ice volume for the three
\ac{AIRs} (paper I). The \ac{AIR} model is freely provided through a git repository for nonprofit purposes
(\url{https://github.com/gayashiva/air_model}, last access: August 1, 2022). However, the model is not expected
to perform well for locations where it has not been calibrated in advance. Specifically, the surface layer
thickness parameter requires prior calibration for better model performance.

\subsection{COSISTUPA: COSIPY + AIR model}
\label{sec:Cosistupa}

To reduce the calibration requirements for the \ac{AIR} model, this was combined with the COupled Snowpack and
Ice surface energy and mass balance model in PYthon (COSIPY). COSIPY is typically used to model distributed snow
and glacier mass changes \citep{sauterCOSIPYV1Opensource2020}. However, it presents a flexible, user-friendly,
and modular framework, making it an ideal platform to implement the alternate modules required for modelling ice
reservoirs. This modified COSIPY model is henceforth referred to as the COSISTUPA model. The COSISTUPA model is
freely provided through a git repository for nonprofit purposes
(\url{https://github.com/cryotools/cosipy/tree/cosistupa}, last access: August 1, 2022). 


\subsubsection{Model configuration}

To convert the COSIPY modules into the COSISTUPA model, the COSIPY model input was extended to include discharge rate and cloudiness index measurements. Additionally, the spray radius parameter was provided as input during model initialization. The model initialization of the ice
cone dimensions was made identical to that of the \ac{AIR} model.

Several parametrizations are available to estimate each surface process in COSIPY. Most of the ones
used in the \ac{AIR} model are among the available options. Additionally, new parametrizations were required to
estimate the conical shape evolution and model the freezing process due to the fountain discharge events. To
extend COSIPY into COSISTUPA, parametrizations of the following processes were modified:

\begin{itemize}

	\item \textbf{Fountain rain heat flux}: The heat flux generated as a result of the difference between the fountain water droplet
	      temperature and the surface temperature was introduced as a new energy balance component. This implementation is
	      identical to the selected approach described in Section \ref{sec:heatflux}.

	\item \textbf{Turbulent flux scaling}: The sensible and the latent heat fluxes were scaled by the $\mu_{cone}$ factor
	      introduced in Section \ref{sec:Qs}.

	\item \textbf{Freezing process}: Phase transition processes were introduced during time periods when the fountain
	      discharge was active. These processes created new ice layers whenever the energy balance allowed it,
	      following the algorithm introduced in Section \ref{sec:phase}.

	\item \textbf{Conical shape evolution}: The surface mass balance estimation was converted to the volume estimation
	      through the methodology described in Section \ref{sec:shape}.

\end{itemize}

Please note the above list of changes is not exhaustive but presents only the major modifications needed to
develop the COSISTUPA model.

\subsubsection{Advantages of the COSISTUPA model}

The COSISTUPA model presents a modular structure that allows the exchange of routines or parametrizations of physical
processes with little effort from the user. The framework consists of a computational kernel, forming
the runtime environment and taking care of the initialization, the input–output routines, and the
parallelization, in addition to the grid and data structures. This structure offers maximum flexibility without
the concern of the internal numerical flow. The adaptive subsurface scheme facilitates an efficient and fast
calculation of the otherwise computationally demanding fundamental equations. The surface energy balance scheme
uses established standard parametrizations for radiation and the energy exchange between atmosphere and surface. The schemes are coupled by solving both surface energy balance and subsurface fluxes
iteratively, with consistent skin temperature being returned at the interface. COSISTUPA uses a one-dimensional
approach limited to the vertical fluxes of energy and matter, neglecting any lateral processes. Accordingly,
the model can be easily set up in parallel computational environments to calculate both energy balance and
climatic surface mass balance of multiple \ac{AIRs} based on flexible horizontal grids and with varying temporal
resolution.

The \ac{RMSE} error of the five ice stupas studied in this thesis is within 20\% of their maximum ice volumes
for the COSISTUPA model (Fig. \ref{fig:Cosistupa}).  However, this model is three times slower than the \ac{AIR} model.
This is likely due to the additional effort required to resolve density and temperature across the two-dimensional model grid.

\begin{figure}[t]
	\centering
	\includegraphics[width=\textwidth]{figs/model_compare.jpg}

	\caption{Comparison of volume estimates generated from the \ac{AIR} and COSISTUPA models. The validation
  dataset derived from drone and \ac{GPR} surveys are represented as dots.}

	\label{fig:Cosistupa}
\end{figure}

The COSISTUPA model is considered to be the best one among the three models for the following reasons:

\begin{itemize}

	\item \textbf{Better spatial temperature and density resolution}: The COSISTUPA model provides temperature and density
	      information of subsurface layers of the \ac{AIR}. In contrast, the \ac{AIR} model computes only the bulk and the
	      surface temperature. Moreover, COSISTUPA is better able to approximate bulk density since it preserves records of
	      previous snowfall events in its subsurface layers.

	\item \textbf{Better parametrization of snowfall}: The densification and albedo decay of snowfall are better
	      handled in COSISTUPA due to its awareness of the snowfall content in each of its subsurface layers.

	\item \textbf{Better validation}: COSISTUPA, being derived from COSIPY, is expected to validate better
	      since the core parametrizations are unchanged and have been extensively validated within other studies \citep{arndtAtmosphereDrivenMassBalance2021}.

	\item \textbf{Future support}: COSISTUPA is expected to be extensively supported in the future by the COSIPY
	      community.

\end{itemize}


